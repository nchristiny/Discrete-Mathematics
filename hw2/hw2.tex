%%%%%%%%%%%%%%%%%%%%%%%%%%%%%%%%%%%%%%%%%
% Sectioned Assignment
% LaTeX Template
% Version 2.0 (Sept. 26 2017)
%
% Heavily modified for use by:
% Nicholas Christiny (http://nchristiny.com/)
% Original author:
% Frits Wenneker (http://www.howtotex.com)
%
% License:
% CC BY-NC-SA 3.0 (http://creativecommons.org/licenses/by-nc-sa/3.0/)
%
%%%%%%%%%%%%%%%%%%%%%%%%%%%%%%%%%%%%%%%%%

%----------------------------------------------------------------------------------------
%	PACKAGES AND OTHER DOCUMENT CONFIGURATIONS
%----------------------------------------------------------------------------------------

\documentclass[11pt, oneside]{article} % A4 paper and 11pt font size

\usepackage[T1]{fontenc} % Use 8-bit encoding that has 256 glyphs
%\usepackage{fourier} % Use the Adobe Utopia font for the document - comment this line to return to the LaTeX default
%\usepackage{fontspec}
%\setmainfont{Brill} % Using Brill http://www.brill.com/about/brill-fonts !!requires LuaTeX!! for error "cannot-use-pdftex", make sure to use LuaTeX (or XeTeX)
\usepackage[english]{babel} % English language/hyphenation
\usepackage{amsmath,amsfonts,amsthm,geometry,amssymb} % Math packages

%\usepackage{sectsty} % Allows customizing section commands
%\allsectionsfont{\scshape} % Make all sections small caps

\usepackage{fancyhdr} % Custom headers and footers
\pagestyle{fancyplain} % Makes all pages in the document conform to the custom headers and footers
\fancyhead{} % No page header - if you want one, create it in the same way as the footers below
\fancyfoot[L]{{\footnotesize{Math146 HW2}}} % Empty left footer
\fancyfoot[C]{} % Empty center footer
\fancyfoot[R]{Page \thepage} % Page numbering for right footer
\renewcommand{\headrulewidth}{0pt} % Remove header underlines
\renewcommand{\footrulewidth}{0pt} % Remove footer underlines
\setlength{\headheight}{13.6pt} % Customize the height of the header

\numberwithin{equation}{section} % Number equations within sections (i.e. 1.1, 1.2, 2.1, 2.2 instead of 1, 2, 3, 4)
\numberwithin{figure}{section} % Number figures within sections (i.e. 1.1, 1.2, 2.1, 2.2 instead of 1, 2, 3, 4)
\numberwithin{table}{section} % Number tables within sections (i.e. 1.1, 1.2, 2.1, 2.2 instead of 1, 2, 3, 4)

%\setlength\parindent{0pt} % Removes all indentation from paragraphs - comment this line for an assignment with lots of text
\setcounter{secnumdepth}{0} % Suppress section numbering
%----------------------------------------------------------------------------------------
%	TITLE SECTION
%----------------------------------------------------------------------------------------

\newcommand{\horrule}[1]{\rule{\linewidth}{#1}} % Create horizontal rule command with 1 argument of height

\title{	
\normalfont \normalsize 
\textsc{Wilbur Wright College, Fall 2017} \\ [25pt] % Your university, school and/or department name(s)
\horrule{0.5pt} \\[0.4cm] % Thin top horizontal rule
\huge Discrete Mathematics Homework II \\
\huge Set Theory, Functions, Algorithms, and Integer Representations \\ % The assignment title
\horrule{2pt} \\[0.5cm] % Thick bottom horizontal rule
}

\author{Nicholas Christiny} % Your name
\date{\normalsize\today} % Today's date or a custom date

\begin{document}

\maketitle % Print the title

%----------------------------------------------------------------------------------------
%----------------------------------------------------------------------------------------

\section{\textsection2.1 Sets}
\subsection{1. List the members of these sets.}
\subsubsection{a) $\{x| x$ is a real number such that $x^2 =1\}$}
Will show the list of members as a set. \\
$\{1, -1\}$
\subsubsection{b) $\{x | x$ is a positive integer less than $12\}$}
$\{1,2,3,4,5,6,7,8,9,10,11\}$
\subsubsection{c) $\{x|x$ is the square of an integer and $x<100\}$}
$\{1,4,9,16,25,36,49,64,81\}$
\subsubsection{d) $\{x|x$ is an integer such that $x^2 =2\}$}
No members. $\{\}$ or $\emptyset$
%----------------------------------------------------------------------------------------

%----------------------------------------------------------------------------------------

\subsection{3. For each of these pairs of sets, determine whether the first is a subset of the second, the second is a subset of the first, or neither is a subset of the other.}
\subsubsection{a) the set of airline flights from New York to New Delhi,\\
the set of nonstop airline flights from New York to New Delhi}
If all airline flights from New York to New Delhi are non-stop, both sets would be equal meaning both would be subsets of each other. However, in the reasonable case that not all flights are nonstop, the second set is a subset of the first and first is not subset of second.
\subsubsection{b) the set of people who speak English, the set of people who speak Chinese}
Neither set is subset of the other.
\subsubsection{c) the set of flying squirrels, the set of living creatures that can fly}
First is subset of second. Second is not subset of first.
%----------------------------------------------------------------------------------------
%----------------------------------------------------------------------------------------

\subsection{5. Determine whether each of these pairs of sets are equal.}
\subsubsection{a) $\{1,3,3,3,5,5,5,5,5\},\{5,3,1\}$}
Equal
\subsubsection{b) $\{\{1\}\},\{1,\{1\}\}$}
Not equal
\subsubsection{c) $\emptyset,\{\emptyset\}$}
Not equal, first is empty set, second is singleton set containing an empty set.
%----------------------------------------------------------------------------------------
%----------------------------------------------------------------------------------------

\subsection{7. For each of the following sets, determine whether 2 is an element of that set.}
\subsubsection{a) $\{x \in \mathbb{R}|x$ is an integer greater than $1\}$}
Yes, 2 is an element of this set.
\subsubsection{b) $\{x \in \mathbb{R}|x$ is the square of an integer$\}$}
No, this set does not contain 2.
\subsubsection{c) $\{2,\{2\}\}$}
Yes.
\subsubsection{d) $\{\{2\},\{\{2\}\}\}$}
No.
\subsubsection{e) $\{\{2\},\{2,\{2\}\}\}$}
No.
\subsubsection{f ) $\{\{\{2\}\}\}$}
No.
%----------------------------------------------------------------------------------------
%----------------------------------------------------------------------------------------

\subsection{9. Determine whether each of these statements is true or false.}
\subsubsection{a) $0\in\emptyset$}
False
\subsubsection{b) $\emptyset\in\{0\}$}
False
\subsubsection{c) $\{0\}\subset\emptyset$}
False
\subsubsection{d) $\emptyset\subset\{0\}$}
False
\subsubsection{e) $\{0\}\in\{0\}$}
True
\subsubsection{f) $\{0\}\subset\{0\}$}
False
\subsubsection{g) $\{\emptyset\}\subseteq\{\emptyset\}$}
True

%----------------------------------------------------------------------------------------

\subsection{11. Determine whether each of these statements is true or false.}
\subsubsection{a) $x\in\{x\}$}
True
\subsubsection{b) $\{x\}\subseteq\{x\}$}
True
\subsubsection{c) $\{x\}\in\{x\}$}
False
\subsubsection{d) $\{x\}\in\{\{x\}\}$}
True
\subsubsection{e) $\emptyset\subseteq\{x\}$}
False
\subsubsection{f) $\emptyset\in\{x\}$}
False

%----------------------------------------------------------------------------------------
%----------------------------------------------------------------------------------------

\subsection{17. Suppose that A, B, and C are sets such that $A \subseteq B$ and $B \subseteq C$. Show that $A \subseteq C$.}
Say set A = \{1, 2\} and set B = \{1, 2, 3\}, where A $\subseteq$ B.\\
Then, set C = \{1, 2, 3, 4, 5, 6\}, such that B $\subseteq$ C.\\
$\therefore$ since \{1, 2\} $\subseteq$ \{1, 2, 3, 4, 5, 6\}, we see A $\subseteq$ C.
 %----------------------------------------------------------------------------------------
%----------------------------------------------------------------------------------------

\subsection{19. What is the cardinality of each of these sets?}
\subsubsection{a) $\{a\}$}
1
\subsubsection{b) $\{\{a\}\}$}
1
\subsubsection{c) $\{a, \{a\}\}$}
2
\subsubsection{d) $\{a, \{a\}, \{a, \{a\}\}\}$}
3
%----------------------------------------------------------------------------------------
%----------------------------------------------------------------------------------------

\subsection{21. Find the power set of each of these sets, where $a$ and $b$ are distinct elements.}
\subsubsection{a) $\{a\}$}
Itself, $\{a\}$
\subsubsection{b) $\{a,b\}$}
$\{a, \{a, b\}\}$
\subsubsection{c) $\{\emptyset,\{\emptyset\}\}$}
$\{\emptyset, \{\emptyset, \{\emptyset\}\}\}$

%----------------------------------------------------------------------------------------
%----------------------------------------------------------------------------------------
\subsection{25. Prove that $\mathcal{P}(A)\subseteq\mathcal{P}(B)$ if and only if $A\subseteq B$.}
...
%----------------------------------------------------------------------------------------
%----------------------------------------------------------------------------------------

\subsection{32. Let $A = \{a,b,c\}$, $B = \{x,y\}$, and $C = \{0,1\}$. Find}
\subsubsection{a) $A \times B \times C$}
$A \times B = \{ (a, x), (a, y), (b, x), (b, y) \}$ \\
$A \times B \times C = \{ ((a, x), 0), ((a, x), 1), ((a, y), 0), ((a, y), 1), ((b, x), 0), ((b, x), 1) \}$\\
...
%----------------------------------------------------------------------------------------
%----------------------------------------------------------------------------------------

\subsection{41. Translate each of these quantifications into English and determine its truth value.}
\subsubsection{a) $\forall x \in \mathbb{R} (x^2 \neq -1)$}
For all x in domain of real numbers, x squared is not equal to negative one. True
\subsubsection{b) $\exists x \in \mathbb{Z} (x^2 = 2)$}
There exists in domain of integers x, such that x squared is two. False
\subsubsection{c) $\forall x \in \mathbb{Z} (x^2 \textgreater 0)$}
For all x in domain of integers, x squared is greater than zero. True
\subsubsection{d) $\exists x \in \mathbb{R} (x^2 = x)$}
There exists in domain of real numbers x, such that x squared equals x. True, $1$ satisfies.
%----------------------------------------------------------------------------------------


\end{document}