%%%%%%%%%%%%%%%%%%%%%%%%%%%%%%%%%%%%%%%%%
% Sectioned Assignment
% LaTeX Template
% Version 2.0 (Sept. 26 2017)
%
% Heavily modified for use by:
% Nicholas Christiny (http://nchristiny.com/)
% Original author:
% Frits Wenneker (http://www.howtotex.com)
%
% License:
% CC BY-NC-SA 3.0 (http://creativecommons.org/licenses/by-nc-sa/3.0/)
%
%%%%%%%%%%%%%%%%%%%%%%%%%%%%%%%%%%%%%%%%%

%----------------------------------------------------------------------------------------
%	PACKAGES AND OTHER DOCUMENT CONFIGURATIONS
%----------------------------------------------------------------------------------------

\documentclass[11pt, oneside]{article} % A4 paper and 11pt font size

\usepackage[T1]{fontenc} % Use 8-bit encoding that has 256 glyphs
%\usepackage{fourier} % Use the Adobe Utopia font for the document - comment this line to return to the LaTeX default
%\usepackage{fontspec}
%\setmainfont{Brill} % Using Brill http://www.brill.com/about/brill-fonts !!requires LuaTeX!! for error "cannot-use-pdftex", make sure to use LuaTeX (or XeTeX)
\usepackage[english]{babel} % English language/hyphenation
\usepackage{amsmath,amsfonts,amsthm,geometry,amssymb} % Math packages

%\usepackage{sectsty} % Allows customizing section commands
%\allsectionsfont{\scshape} % Make all sections small caps

\usepackage{fancyhdr} % Custom headers and footers
\pagestyle{fancyplain} % Makes all pages in the document conform to the custom headers and footers
\fancyhead{} % No page header - if you want one, create it in the same way as the footers below
\fancyfoot[L]{{\footnotesize{Math146 HW2}}} % Empty left footer
\fancyfoot[C]{} % Empty center footer
\fancyfoot[R]{Page \thepage} % Page numbering for right footer
\renewcommand{\headrulewidth}{0pt} % Remove header underlines
\renewcommand{\footrulewidth}{0pt} % Remove footer underlines
\setlength{\headheight}{13.6pt} % Customize the height of the header

\numberwithin{equation}{section} % Number equations within sections (i.e. 1.1, 1.2, 2.1, 2.2 instead of 1, 2, 3, 4)
\numberwithin{figure}{section} % Number figures within sections (i.e. 1.1, 1.2, 2.1, 2.2 instead of 1, 2, 3, 4)
\numberwithin{table}{section} % Number tables within sections (i.e. 1.1, 1.2, 2.1, 2.2 instead of 1, 2, 3, 4)

%\setlength\parindent{0pt} % Removes all indentation from paragraphs - comment this line for an assignment with lots of text
\setcounter{secnumdepth}{0} % Suppress section numbering
%----------------------------------------------------------------------------------------
%	TITLE SECTION
%----------------------------------------------------------------------------------------

\newcommand{\horrule}[1]{\rule{\linewidth}{#1}} % Create horizontal rule command with 1 argument of height

\title{
\normalfont \normalsize
\textsc{Wilbur Wright College, Fall 2017} \\ [25pt] % Your university, school and/or department name(s)
\horrule{0.5pt} \\[0.4cm] % Thin top horizontal rule
\huge Discrete Mathematics Homework II \\
\huge Set Theory, Functions, Algorithms, and Integer Representations \\ % The assignment title
\horrule{2pt} \\[0.5cm] % Thick bottom horizontal rule
}

\author{Nicholas Christiny} % Your name
\date{\normalsize\today} % Today's date or a custom date

\begin{document}

\maketitle % Print the title

%----------------------------------------------------------------------------------------
%----------------------------------------------------------------------------------------

\section{\textsection2.1 Sets}
\subsection{1. List the members of these sets.}
\subsubsection{a) $\{x| x$ is a real number such that $x^2 =1\}$}
Will show the list of members as a set. \\
$\{1, -1\}$
\subsubsection{b) $\{x | x$ is a positive integer less than $12\}$}
$\{1,2,3,4,5,6,7,8,9,10,11\}$
\subsubsection{c) $\{x|x$ is the square of an integer and $x<100\}$}
$\{1,4,9,16,25,36,49,64,81\}$
\subsubsection{d) $\{x|x$ is an integer such that $x^2 =2\}$}
No members. $\{\}$ or $\emptyset$
%----------------------------------------------------------------------------------------

%----------------------------------------------------------------------------------------

\subsection{3. For each of these pairs of sets, determine whether the first is a subset of the second, the second is a subset of the first, or neither is a subset of the other.}
\subsubsection{a) the set of airline flights from New York to New Delhi,\\
the set of nonstop airline flights from New York to New Delhi}
If all airline flights from New York to New Delhi are non-stop, both sets would be equal meaning both would be subsets of each other. However, in the reasonable case that not all flights are nonstop, the second set is a subset of the first and first is not subset of second.
\subsubsection{b) the set of people who speak English, the set of people who speak Chinese}
Neither set is subset of the other.
\subsubsection{c) the set of flying squirrels, the set of living creatures that can fly}
First is subset of second. Second is not subset of first.
%----------------------------------------------------------------------------------------
%----------------------------------------------------------------------------------------

\subsection{5. Determine whether each of these pairs of sets are equal.}
\subsubsection{a) $\{1,3,3,3,5,5,5,5,5\},\{5,3,1\}$}
Equal
\subsubsection{b) $\{\{1\}\},\{1,\{1\}\}$}
Not equal
\subsubsection{c) $\emptyset,\{\emptyset\}$}
Not equal, first is empty set, second is singleton set containing an empty set.
%----------------------------------------------------------------------------------------
%----------------------------------------------------------------------------------------

\subsection{7. For each of the following sets, determine whether 2 is an element of that set.}
\subsubsection{a) $\{x \in \mathbb{R}|x$ is an integer greater than $1\}$}
Yes, 2 is an element of this set.
\subsubsection{b) $\{x \in \mathbb{R}|x$ is the square of an integer$\}$}
No, this set does not contain 2.
\subsubsection{c) $\{2,\{2\}\}$}
Yes.
\subsubsection{d) $\{\{2\},\{\{2\}\}\}$}
No.
\subsubsection{e) $\{\{2\},\{2,\{2\}\}\}$}
No.
\subsubsection{f ) $\{\{\{2\}\}\}$}
No.
%----------------------------------------------------------------------------------------
%----------------------------------------------------------------------------------------

\subsection{9. Determine whether each of these statements is true or false.}
\subsubsection{a) $0\in\emptyset$}
False
\subsubsection{b) $\emptyset\in\{0\}$}
False
\subsubsection{c) $\{0\}\subset\emptyset$}
False
\subsubsection{d) $\emptyset\subset\{0\}$}
False
\subsubsection{e) $\{0\}\in\{0\}$}
True
\subsubsection{f) $\{0\}\subset\{0\}$}
False
\subsubsection{g) $\{\emptyset\}\subseteq\{\emptyset\}$}
True

%----------------------------------------------------------------------------------------

\subsection{11. Determine whether each of these statements is true or false.}
\subsubsection{a) $x\in\{x\}$}
True
\subsubsection{b) $\{x\}\subseteq\{x\}$}
True
\subsubsection{c) $\{x\}\in\{x\}$}
False
\subsubsection{d) $\{x\}\in\{\{x\}\}$}
True
\subsubsection{e) $\emptyset\subseteq\{x\}$}
False
\subsubsection{f) $\emptyset\in\{x\}$}
False

%----------------------------------------------------------------------------------------
%----------------------------------------------------------------------------------------

\subsection{17. Suppose that A, B, and C are sets such that $A \subseteq B$ and $B \subseteq C$. Show that $A \subseteq C$.}
Say set A = \{1, 2\} and set B = \{1, 2, 3\}, where A $\subseteq$ B.\\
Then, set C = \{1, 2, 3, 4, 5, 6\}, such that B $\subseteq$ C.\\
$\therefore$ since \{1, 2\} $\subseteq$ \{1, 2, 3, 4, 5, 6\}, we see A $\subseteq$ C.
 %----------------------------------------------------------------------------------------
%----------------------------------------------------------------------------------------

\subsection{19. What is the cardinality of each of these sets?}
\subsubsection{a) $\{a\}$}
1
\subsubsection{b) $\{\{a\}\}$}
1
\subsubsection{c) $\{a, \{a\}\}$}
2
\subsubsection{d) $\{a, \{a\}, \{a, \{a\}\}\}$}
3
%----------------------------------------------------------------------------------------
%----------------------------------------------------------------------------------------

\subsection{21. Find the power set of each of these sets, where $a$ and $b$ are distinct elements.}
\subsubsection{a) $\{a\}$}
Itself, $\{a\}$
\subsubsection{b) $\{a,b\}$}
$\{a, \{a, b\}\}$
\subsubsection{c) $\{\emptyset,\{\emptyset\}\}$}
$\{\emptyset, \{\emptyset, \{\emptyset\}\}\}$

%----------------------------------------------------------------------------------------
%----------------------------------------------------------------------------------------
\subsection{25. Prove that $\mathcal{P}(A)\subseteq\mathcal{P}(B)$ if and only if $A\subseteq B$.}
...
%----------------------------------------------------------------------------------------
%----------------------------------------------------------------------------------------

\subsection{32. Let $A = \{a,b,c\}$, $B = \{x,y\}$, and $C = \{0,1\}$. Find}
\subsubsection{a) $A \times B \times C$}
$A \times B = \{ (a, x), (a, y), (b, x), (b, y) \}$ \\
$A \times B \times C = \{ ((a, x), 0), ((a, x), 1), ((a, y), 0), ((a, y), 1), ((b, x), 0), ((b, x), 1) \}$\\
...
%----------------------------------------------------------------------------------------
%----------------------------------------------------------------------------------------

\subsection{41. Translate each of these quantifications into English and determine its truth value.}
\subsubsection{a) $\forall x \in \mathbb{R} (x^2 \neq -1)$}
For all x in domain of real numbers, x squared is not equal to negative one. True
\subsubsection{b) $\exists x \in \mathbb{Z} (x^2 = 2)$}
There exists in domain of integers x, such that x squared is two. False
\subsubsection{c) $\forall x \in \mathbb{Z} (x^2 \textgreater 0)$}
For all x in domain of integers, x squared is greater than zero. True
\subsubsection{d) $\exists x \in \mathbb{R} (x^2 = x)$}
There exists in domain of real numbers x, such that x squared equals x. True, $1$ satisfies.
%----------------------------------------------------------------------------------------
%----------------------------------------------------------------------------------------

\section{\textsection2.2}
\subsection{3. Let $A = \{1,2,3,4,5\}$ and $B = \{0,3,6\}$. Find}
\subsubsection{a) $A\cup B$}
$\{0,1,2,3,4,5,6 \}$
\subsubsection{b) $A\cap B$}
$\{3\}$
\subsubsection{c) $A-B$}
$\{1,2,4,5\}$
\subsubsection{d) $B-A$}
$\{0,6\}$
%----------------------------------------------------------------------------------------
%----------------------------------------------------------------------------------------

\subsection{19. Show that if A and B are sets, then}
\subsubsection{a) $A-B=A\cap \overline{B}$}
%---draw some Venn diagrams... old school style - by hand
Venn Diagram: \\\\\\\\\\\
\subsubsection{b) $(A\cap B)\cup (A\cap \overline{B})=A$}
%---draw some Venn diagrams... old school style - by hand
Venn Diagram: \\\\\\\\\\\
%----------------------------------------------------------------------------------------
%----------------------------------------------------------------------------------------

\subsection{25. Let $A = \{0,2,4,6,8,10\}$, $B = \{0,1,2,3,4,5,6\}$, and
$C = \{4,5,6,7,8,9,10\}$. Find}
\subsubsection{a) $A\cap B\cap C$}
$\{0,2,4,6\}\cap \{4,5,6,7,8,9,10\} = \{4,6\}$
\subsubsection{b) $A\cup B\cup C$}
$\{0,1,2,3,4,5,6,8,10\} \cup \{4,5,6,7,8,9,10\} = \{0,1,2,3,4,5,6,7,8,9,10\}$
\subsubsection{c) $(A\cup B)\cap C$}
$\{0,1,2,3,4,5,6,8,10\} \cap \{4,5,6,7,8,9,10\} = \{4,5,6,8,10\}$
\subsubsection{d) $(A\cap B)\cup C$}
$\{0,2,4,6\} \cup \{4,5,6,7,8,9,10\} = \{0,2,4,5,6,7,8,9,10\}$
%----------------------------------------------------------------------------------------
%----------------------------------------------------------------------------------------

\subsection{27. Draw the Venn diagrams for each of these combinations of the sets A, B, and C.}
\subsubsection{a) $A\cap (B-C)$}
Venn Diagram: \\\\\\\\
\subsubsection{b) $(A\cap B)\cup (A\cap C)$}
Venn Diagram: \\\\\\\\
\subsubsection{c) $(A\cap B)\cup (A\cap\overline{C})$}
Venn Diagram: \\\\\\
%----------------------------------------------------------------------------------------
%----------------------------------------------------------------------------------------

\subsection{32. Find the symmetric difference of $\{1, 3, 5\}$ and $\{1, 2, 3\}$}
$\{2,5\}$

%----------------------------------------------------------------------------------------
%----------------------------------------------------------------------------------------

\subsection{37. Show that if A is a subset of a universal set U , then}
\subsubsection{a) $A\oplus A=\emptyset$}
Venn Diagram: \\\\\\\\
\subsubsection{b) $A\oplus \emptyset=A$}
Venn Diagram: \\\\\\\\
\subsubsection{c) $A\oplus U =\overline{A}$}
Venn Diagram: \\\\\\\\
\subsubsection{d) $A\oplus \overline{A} = U$}
Venn Diagram: \\\\\\\\


%----------------------------------------------------------------------------------------
%----------------------------------------------------------------------------------------

\section{\textsection2.3}
\subsection{1. Why is $f$ not a function from $\mathbb{R}$ to $\mathbb{R}$ if}
\subsubsection{a) $f(x)=1/x$?}
Because f(zero) is dividing by zero.
\subsubsection{b) $f(x)= \sqrt x$?}
Because any negative value such as x = -1, gives $\sqrt -1$, an imaginary number part of complex set $\mathbb{C}$.
\subsubsection{c) $f(x)=\pm \sqrt{(x^2+1)}$?}
There are two possible values for every f(x) - not a function by definition.
%----------------------------------------------------------------------------------------
%----------------------------------------------------------------------------------------

\subsection{9. Find these values.}
\subsubsection{a) $\lceil \frac{3}{4}\rceil$}
1
\subsubsection{b) $\lfloor \frac{7}{8}\rfloor $}
0
\subsubsection{c) $\lceil -\frac{3}{4}\rceil$}
0
\subsubsection{d) $\lfloor -\frac{7}{8}\rfloor$}
-1
\subsubsection{e) $\lceil 3\rceil$}
3
\subsubsection{f) $\lfloor -1\rfloor$}
-1
\subsubsection{g) $\lfloor \frac{1}{2}+\lceil \frac{3}{2}\rceil\rfloor$}
2
\subsubsection{h) $\lfloor \frac{1}{2}\cdot \lfloor \frac{5}{2}\rfloor\rfloor$}
1

%----------------------------------------------------------------------------------------
%----------------------------------------------------------------------------------------

\subsection{10. Determine whether each of these functions from $\{a, b, c, d\}$ to itself is one-to-one.}
\subsubsection{a) $f(a)=b,f(b)=a,f(c)=c,f(d)=d$}
One-to-one
\subsubsection{b) $f(a)=b,f(b)=b,f(c)=d,f(d)=c$}
Not one-to-one
\subsubsection{c) $f(a)=d,f(b)=b,f(c)=c,f(d)=d$}
Not one-to-one
%----------------------------------------------------------------------------------------
%----------------------------------------------------------------------------------------

\subsection{11. Which functions in Exercise 10 are onto? }
Only a. is onto function.

%----------------------------------------------------------------------------------------
%----------------------------------------------------------------------------------------
\subsection{23. Determine whether each of these functions is a bijection from $\mathbb{R}$ to $\mathbb{R}$.}
\subsubsection{a) $f(x)=2x+1$}
One-to-one and onto, Yes bijection.
\subsubsection{b) $f(x)=x^2+1$}
Not one-to-one, so no bijection.
\subsubsection{c) $f(x)=x^3$}
Not one-to-one, not bijective.
\subsubsection{d) $f(x)=(x^2 +1)/(x^2 +2)$}
...
%----------------------------------------------------------------------------------------
%----------------------------------------------------------------------------------------

\subsection{32. Let $f(x) = 2x$ where the domain is the set of real numbers. What is}
\subsubsection{a) $f(\mathbb{Z})$?}
\subsubsection{b) $f(\mathbb{N})$?}
\subsubsection{c) $f(\mathbb{R})$?}
...

%----------------------------------------------------------------------------------------
%----------------------------------------------------------------------------------------

\subsection{33. Suppose that $g$ is a function from A to B and $f$ is a function from B to C.}
\subsubsection{a) Show that if both $f$ and $g$ are one-to-one functions, then $f\circ g$ is also one-to-one.}
\subsubsection{b) Show that if both $f$ and $g$ are onto functions, then $f\circ g$ is also onto.}
%----------------------------------------------------------------------------------------
%----------------------------------------------------------------------------------------

\subsection{45. Let $f$ be a function from A to B. Let S be a subset of B. Show that $f^-1(S)=\overline{f^-1(S)}$.}
...
%----------------------------------------------------------------------------------------
%----------------------------------------------------------------------------------------

\section{\textsection3.1}
\subsection{1. List all the steps used by Algorithm 1 to find the maximum of the list 1, 8, 12, 9, 11, 2, 14, 5, 10, 4.}
Make first element max, max equals 1.\\
For all subsequent elements, if that element is higher than max, replace max.\\
Step 2, 8 replaces 1 as max. \\
Step 3, 12 becomes new max.\\
Step 4, nothing happens since 9 $\textless$ 12\\
Same for steps 5 and 6 but at step 7, 14 $\textgreater$ 12 so max becomes 14.\\
Iterate and check rest of elements and \\
return max when all elements have been checked.

%----------------------------------------------------------------------------------------
%----------------------------------------------------------------------------------------

\subsection{3. Devise an algorithm that finds the sum of all the integers in a list.}
\begin{tabbing}
Procedure: Sum(L: list of n integers) \\
\=  total := o \= \= \= \= \= \\
\> for i:= 1 to n\\
\> \> total = Li + total \\
\> return total
\end{tabbing}
%----------------------------------------------------------------------------------------
%----------------------------------------------------------------------------------------

\subsection{4. Describe an algorithm that takes as input a list of $n$ integers and produces as output the largest difference obtained by subtracting an integer in the list from the one following it.}
\begin{tabbing}
Procedure: BigDiff(L: list of n integers) \\
\=  bigg\=estDiff \=:= o \= \= \= \= \\
for i:= 2 to (n-1) \\
\> \> if biggestDiff less than abs(L\textsubscript{i-1} - L\textsubscript{i}) \\
\> \> \> Replace biggestDiff with value \\
\> return biggestDiff
\end{tabbing}

%----------------------------------------------------------------------------------------
\subsubsection{6. Describe an algorithm that takes as input a list of n integers and finds the number of negative integers in the list.}
\begin{tabbing}
Procedure: NumberOfNegatives(L: list of n integers) \\
\=  count := o \= \= \= \= \= \\
\> for i:= 1 to n\\
\> \> if L\textsubscript{i} less than 0 \\
\> \> \> \> count = count + 1 \\
\> return count
\end{tabbing}

\subsubsection{9. A palindrome is a string that reads the same forward and backward. Describe an algorithm for determining whether a string of n characters is a palindrome.}
\begin{tabbing}
Procedure: PalidromeDetect(text: string) \\
\= $\quad$ \= $\quad$ \= $\quad$ \= $\quad$ \= $\quad$ \= $\quad$\\
if text is "" \\
\> \> return false \\
left:= 0 \\
right := text.length - 1 \\
while (left < right) \\
\> \> if text[left] not equal to text[right] \\
\> \> \> \>return false \\
\> \>left:= left + 1 \\
\> \>right right - 1 \\\\
return true
\end{tabbing}

%----------------------------------------------------------------------------------------

\subsection{13. List all the steps used to search for 9 in the sequence 1, 3, 4, 5, 6, 8, 9, 11 using}
\subsubsection{a) a linear search.}
Set a zero or 1-based index counter \\
check first value, 1. It is not target, 9. \\
increment index counter \\
Move on, check second term 3, not 9. \\
increment index counter \\
Third term 4, not 9 move on, \\
increment index counter \\
check fourth term 5, not 9, move to next, \\
increment index counter \\
check 6, not 9 move on to next, \\
increment index counter \\
check value of 8 not 9, move to next value \\
increment index counter \\
check value of 9 is same as target 9 \\
return index counter for position of target
\subsubsection{b) a binary search.}
Set some initial counters to zero or 1 per preference \\
Divide ordered sequence in two,
check value versus top index of lower half \\
For our sequence search, 5 < 9, so we take top half of sequence and reassign indices, to keep track of original positions \\
We continue to bisect, 6,8 and 9,11 \\
check target versus top of bottom bisection shows 8 < 9, so we repeat bisection of top half.\\
this gives 8 and 9, checking bottom half versus target shows 8 < 9, \\
the algorithm will continue to bisect leaving just 9 \\
a final check shows 9's index position that can be traced back using the index counter, position 7 for 1-based index \\\\\\\\\\
%----------------------------------------------------------------------------------------
%----------------------------------------------------------------------------------------

\subsection{20. Describe an algorithm for finding both the largest and the smallest integers in a finite sequence of integers.}
\begin{tabbing}
Procedure: BigSmall(L: list of n integers) \\
\=  max := o \= \= \= \= \= \\
min := o
for i:= 1 to n\\
\> \> if max less than L\textsubscript{n} \\
\> \>\> \> max := L\textsubscript{n} \\
\> \> if min greater than L\textsubscript{n} \\
\> \>\> \> min := L\textsubscript{n} \\
return (max, min)
\end{tabbing}
%----------------------------------------------------------------------------------------
%----------------------------------------------------------------------------------------

\subsection{27. The ternary search algorithm }
...
%----------------------------------------------------------------------------------------
%----------------------------------------------------------------------------------------

\subsection{7. Adapt the bubble sort algorithm so that it stops when no interchanges are required. Express this more efficient version of the algorithm in pseudocode.}
...
%----------------------------------------------------------------------------------------
%----------------------------------------------------------------------------------------

\section{\textsection4.2}
\subsection{1. Convert the decimal expansion of each of these integers to a binary expansion.}
\subsubsection{a) 231}
(Please note work for these shown separately at end of HW) \\
(1110 0111)$\textsubscript{2}$ \\
\subsubsection{b) 4532}
(1 0001 1011 0100)$\textsubscript{2}$
%----------------------------------------------------------------------------------------

%----------------------------------------------------------------------------------------

\subsection{2. Convert the decimal expansion of each of these integers to a binary expansion.}
\subsubsection{a) 321}
(1 0000 0101)$\textsubscript{2}$
\subsubsection{b) 1023}
(11 1111 1111)$\textsubscript{2}$
%----------------------------------------------------------------------------------------

%----------------------------------------------------------------------------------------

\subsection{3. Convert the binary expansion of each of these integers to a decimal expansion.}
\subsubsection{a) (1 1111)$\textsubscript{2}$}
31
\subsubsection{b) (10 0000 0001)$\textsubscript{2}$}
513
\subsubsection{c) (1 0101 0101)$\textsubscript{2}$}
341
\subsubsection{d) (110 1001 0001 0000)$\textsubscript{2}$}
26,897

%----------------------------------------------------------------------------------------
%----------------------------------------------------------------------------------------

\subsection{5. Convert the octal expansion of each of these integers to a
binary expansion.}
\subsubsection{b) (1604)$\textsubscript{8}$}
(1110 0100)$\textsubscript{2}$
\subsubsection{d) (2417)$\textsubscript{8}$}
(1 0100 1111)$\textsubscript{2}$
%----------------------------------------------------------------------------------------
%----------------------------------------------------------------------------------------

\subsection{6. Convert the binary expansion of each of these integers to an octal expansion.}
\subsubsection{b) (1010 1010 1010)$\textsubscript{2}$}
(5252)$\textsubscript{8}$
\subsubsection{d) (101 0101 0101 0101)$\textsubscript{2}$}
(52525)$\textsubscript{8}$
%----------------------------------------------------------------------------------------

%----------------------------------------------------------------------------------------

\subsection{7. Convert the hexadecimal expansion of each of these integers to a binary expansion}
\subsubsection{b) (135AB)$\textsubscript{16}$}
(1 00110101 1010 1011)$\textsubscript{2}$
\subsubsection{d) (DEFACED)16}
(1101 1110 1111 1010 1100 1110 1101)$\textsubscript{2}$
%----------------------------------------------------------------------------------------

%----------------------------------------------------------------------------------------

\subsection{11. Convert (1011 0111 1011)$\textsubscript{2}$ from its binary expansion to its hexadecimal expansion.}
(B7B)$\textsubscript{16}$

%----------------------------------------------------------------------------------------
%----------------------------------------------------------------------------------------

\subsection{12. Convert (1 1000 0110 0011)$\textsubscript{2}$ from its binary expansion to its hexadecimal expansion.}
(1863)$\textsubscript{16}$
%----------------------------------------------------------------------------------------

\end{document}
