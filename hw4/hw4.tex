%%%%%%%%%%%%%%%%%%%%%%%%%%%%%%%%%%%%%%%%%
% Sectioned Assignment
% LaTeX Template
% Version 2.0 (Sept. 26 2017)
%
% Heavily modified for use by:
% Nicholas Christiny (http://nchristiny.com/)
% Original author:
% Frits Wenneker (http://www.howtotex.com)
%
% License:
% CC BY-NC-SA 3.0 (http://creativecommons.org/licenses/by-nc-sa/3.0/)
%
%%%%%%%%%%%%%%%%%%%%%%%%%%%%%%%%%%%%%%%%%

%----------------------------------------------------------------------------------------
% PACKAGES AND OTHER DOCUMENT CONFIGURATIONS
%----------------------------------------------------------------------------------------

\documentclass[11pt, oneside]{article} % A4 paper and 11pt font size

\usepackage[T1]{fontenc} % Use 8-bit encoding that has 256 glyphs
%\usepackage{fourier} % Use the Adobe Utopia font for the document - comment this line to return to the LaTeX default
\usepackage{fontspec}
%\setmainfont{Brill} % Using Brill http://www.brill.com/about/brill-fonts !!requires LuaTeX!! for error "cannot-use-pdftex", make sure to use LuaTeX (or XeTeX)

\usepackage[english]{babel} % English language/hyphenation
\usepackage{amsmath,amsfonts,amsthm,geometry,amssymb} % Math packages

%\usepackage{sectsty} % Allows customizing section commands
%\allsectionsfont{\scshape} % Make all sections small caps

\usepackage{fancyhdr} % Custom headers and footers
\pagestyle{fancyplain} % Makes all pages in the document conform to the custom headers and footers
\fancyhead{} % No page header - if you want one, create it in the same way as the footers below
\fancyfoot[L]{{\footnotesize{Math146 HW3}}} % Empty left footer
\fancyfoot[C]{} % Empty center footer
\fancyfoot[R]{Page \thepage} % Page numbering for right footer
\renewcommand{\headrulewidth}{0pt} % Remove header underlines
\renewcommand{\footrulewidth}{0pt} % Remove footer underlines
\setlength{\headheight}{13.6pt} % Customize the height of the header

\numberwithin{equation}{section} % Number equations within sections (i.e. 1.1, 1.2, 2.1, 2.2 instead of 1, 2, 3, 4)
\numberwithin{figure}{section} % Number figures within sections (i.e. 1.1, 1.2, 2.1, 2.2 instead of 1, 2, 3, 4)
\numberwithin{table}{section} % Number tables within sections (i.e. 1.1, 1.2, 2.1, 2.2 instead of 1, 2, 3, 4)

%\setlength\parindent{0pt} % Removes all indentation from paragraphs - comment this line for an assignment with lots of text
\setcounter{secnumdepth}{0} % Suppress section numbering
%----------------------------------------------------------------------------------------
% TITLE SECTION
%----------------------------------------------------------------------------------------

\newcommand{\horrule}[1]{\rule{\linewidth}{#1}} % Create horizontal rule command with 1 argument of height

\title{
\normalfont \normalsize
\textsc{Wilbur Wright College, Fall 2017} \\ [25pt] % Your university, school and/or department name(s)
\horrule{0.5pt} \\[0.4cm] % Thin top horizontal rule
\huge Discrete Mathematics Homework IV\\ % The assignment title
\huge Relations \\
\horrule{2pt} \\[0.5cm] % Thick bottom horizontal rule
}

\author{Nicholas Christiny} % Your name
\date{\normalsize\today} % Today's date or a custom date

\begin{document}

\maketitle % Print the title

%----------------------------------------------------------------------------------------
%----------------------------------------------------------------------------------------
%----------------------------------------------------------------------------------------

\section{\textsection9.1}
\subsection{1. List the ordered pairs in the relation $R$ from $A = {0,1,2,3,4}$ to $B = {0,1,2,3}$, where $(a,b) ∈ R$ if and only if}
\subsubsection{a) $a=b$.}
$$R = \{(0,0),(1,1),(2,2),(3,3)\}$$
\subsubsection{b) $a+b=4$.}
$$R=\{(1,3),(2,2),(3,1),(4,0)\}$$
\subsubsection{c) $a>b$.}
$$R=\{(1,0),(2,0),(2,1),(3,0),(3,1),(3,2),(4,0),(4,1),(4,2),(4,3)\}$$
\subsubsection{d) $a|b$.}
$$ R=\{(1,0),(1,1),(1,2),(1,3),(2,0),(2,2),(3,0),(3,3),(4,0)\}  $$
%----------------------------------------------------------------------------------------
%----------------------------------------------------------------------------------------

\subsection{3. For each of these relations on the set $\{1, 2, 3, 4\}$, decide whether it is reflexive, whether it is symmetric, whether it is antisymmetric, and whether it is transitive.}
\subsubsection{a) $\{(2,2),(2,3),(2,4),(3,2),(3,3),(3,4)\}$}
Reflexive, no. Symmetric, no. Anti-symmetric, no, more than one edge connects 2-points. Transitive? Yes.
\subsubsection{b) $\{(1, 1), (1, 2), (2, 1), (2, 2), (3, 3), (4, 4)\}$}
Reflexive? Yes Symmetric? Yes Anti-symmetric? No. Transitive? Yes.
\subsubsection{c) $\{(2, 4), (4, 2)\}$}
Reflexive? No. Symmetric? Yes. Anti-symmetric? No. Transitive? No.
\subsubsection{d) $\{(1, 2), (2, 3), (3, 4)\}$}
Reflexive? No. Symmetric? No. Anti-symmetric? Yes. Transitive? No.
\subsubsection{e) $\{(1, 1), (2, 2), (3, 3), (4, 4)\}$}
Reflexive? Yes. Symmetric? Yes. Anti-symmetric? Yes. Transitive? Yes.
\subsubsection{f) $\{(1, 3), (1, 4), (2, 3), (2, 4), (3, 1), (3, 4)\}$}
Reflexive? No.Symmetric? No. Anti-symmetric? No. Transitive? No. None.
%----------------------------------------------------------------------------------------

\subsection{6. Determine whether the relation $R$ on the set of all real numbers is reflexive, symmetric, antisymmetric, and/or transitive, where $(x , y ) ∈ R$ if and only if}
\subsubsection{a)$x+y=0$.}
Reflexive, yes.Symmetric, yes. Anti-symmetric, no. Transitive, yes.
\subsubsection{c) $x−y$ is a rational number.}
...
\subsubsection{e) $x\cdot y\geq 0$.}
Reflexive, no. Symmetric, yes. Anti-symmetric, no. Transitive, yes.

\subsubsection{h) $x=1$ or $y=1$.}
Reflexive, no. Symmetric, no. Anti-symmetric, yes. Transitive,  yes?
%----------------------------------------------------------------------------------------
%----------------------------------------------------------------------------------------

\subsection{7. Determine whether the relation $R$ on the set of all integers is reflexive, symmetric, antisymmetric, and/or transitive, where $(x,y) \in R$ if and only if}
\subsubsection{b.) $xy \geq 1$}
Reflexive? No. Symmetric, Yes. Anti-symmetric, No. Transitive, Yes.
\subsubsection{c) $x=y+1$ or $x=y−1$.}
Reflexive, No. Symmetric, Yes. Anti-symmetric, No. Transitive, No.
\subsubsection{d) $x \equiv y(mod 7)$.}
Reflexive, No. Symmetric, No. Anti-symmetric, Yes. Transitive, No.
\subsubsection{g) $x=y^2$.}
Reflexive, No. Symmetric, No. Anti-symmetric, Yes. Transitive, No.

%----------------------------------------------------------------------------------------

%----------------------------------------------------------------------------------------
%----------------------------------------------------------------------------------------

\section{\textsection9.5}
\subsection{1. Which of these relations on $\{0, 1, 2, 3\}$ are equivalence relations? Determine the properties of an equivalence relation that the others lack.}
\subsubsection{a) $\{(0,0),(1,1),(2,2),(3,3)\}$}
Reflexive? Yes. Symmetric, Yes. Transitive, Yes. Yes, it is an equivalence relation.
\subsubsection{b) $\{(0,0),(0,2),(2,0),(2,2),(2,3),(3,2),(3,3)\}$}
Reflexive, No. Not transitive. Not an equivalence relation.
\subsubsection{c) $\{(0,0),(1,1),(1,2),(2,1),(2,2),(3,3)\}$}
Reflexive, yes. Symmetric, yes. Transitive, Yes. It is an equivalence relation.
\subsubsection{d) $\{(0,0),(1,1),(1,3),(2,2),(2,3),(3,1),(3,2),(3, 3)\}$}
Reflexive, Yes. Symmetric, yes. Transitive, No. Not an equivalence relation.
\subsubsection{e) $\{(0, 0), (0, 1), (0, 2), (1, 0), (1, 1), (1, 2), (2, 0),
(2, 2), (3, 3)\}$}
Reflexive, yes. Symmetric, no. Transitive, No. Not an equivalence relation.



%----------------------------------------------------------------------------------------
%----------------------------------------------------------------------------------------

\subsection{7. Show that the relation of logical equivalence on the set of all compound propositions is an equivalence relation. What are the equivalence classes of F and T?}
Logically equivalent propositions have the same truth tables. p logically equivalent to q is reflexive, because p is logically equivalent to p; it is also symmetric, since statement p is logically equivalent to q is the same as q is logically equivalent to p; and while p is logically equivalent to q, if q is logically equivalent to r, then p is logically equivalent to r as well, which means transitive property holds. Therefore, it is an equivalence relation.\\
The equivalence class of T is the set of all propositions that are tautologies.   \\
The equivalence class of F is the set of all propositions that are contradictions. \\

%----------------------------------------------------------------------------------------

\subsection{15. Let R be the relation on the set of ordered pairs of positive integers such that $((a, b), (c, d)) \in R$ if and only if $a + d = b + c$. Show that $R$ is an equivalence relation.}
Reflexive? $((a,b),(b,a)) \in R$, because $a+b = b +a$ \\
Symmetric? $a+b = b +c$ so $c+b = d+a$\\
Transitive? $a+d=b+c$, and $c + e = d +f$, so $a+d+c+e = b+c+d+f$, so $a+e=b+f$
%----------------------------------------------------------------------------------------
%----------------------------------------------------------------------------------------

\subsection{In Exercises 21–23 determine whether the relation with the
directed graph shown is an equivalence relation.}
%----------------------------------------------------------------------------------------
\subsection{21.}
Reflexive? Yes. Symmetric? Yes. Transitive? No. Not an equivalence relation.

%----------------------------------------------------------------------------------------
\subsection{23.}
Reflexive? Yes. Symmetric? Yes. Transitive? No. Not an equivalence relation.

%----------------------------------------------------------------------------------------

\subsection{35. What is the congruence class $[n]5$ (that is, the equivalence class of $n$ with respect to congruence modulo 5) when $n$ is}
\subsubsection{a) 2?}
$[a]_m = \{ \ldots, a-2m,a-m,a,a+m,a+2m, \ldots \}$ \\
$=\{ \ldots, -8, -3, 2, 7, 12, \ldots \}$
\subsubsection{b) 3?}
$=\{ \ldots, -7, -2, 3, 8, 13, \ldots \}$
\subsubsection{c) 6?}
$=\{ \ldots, -4, 1, 6, 11, 16, \ldots \}$
\subsubsection{d) −3?}
$=\{ \ldots, -13, -8, -3, 2, 7, \ldots \}$

%----------------------------------------------------------------------------------------
%----------------------------------------------------------------------------------------

\subsection{39.}
\subsubsection{a) What is the equivalence class of $(1, 2)$ with respect to the equivalence relation in Exercise 15?}
...
\subsubsection{b) Give an interpretation of the equivalence classes for the equivalence relation R in Exercise 15. [Hint: Look at the difference a − b corresponding to (a, b).]}
...
%----------------------------------------------------------------------------------------
%----------------------------------------------------------------------------------------

\section{\textsection9.6}
\subsection{1. Which of these relations on $\{0, 1, 2, 3\}$ are partial orderings? Determine the properties of a partial ordering that the others lack.}
\subsubsection{a) $\{(0, 0), (1, 1), (2, 2), (3, 3)\}$}
Reflexive, yes. Anti-symmetric, yes. Transitive, yes proven earlier. Partial ordering confirmed.
\subsubsection{b) $\{(0, 0), (1, 1), (2, 0), (2, 2), (2, 3), (3, 2), (3, 3)\}$}
Reflexive, yes. Anti-symmetric, no. Not transitive. Not a partial ordering.
\subsubsection{c) $\{(0, 0), (1, 1), (1, 2), (2, 2), (3, 3)\}$}
Reflexive, yes. Anti-symmetric, yes. Transitive, yes. Partial ordering.
\subsubsection{d) $\{(0, 0), (1, 1), (1, 2), (1, 3), (2, 2), (2, 3), (3, 3)\}$}
Reflexive, yes. Anti-symmetric, yes. Transitive, yes. Partial ordering.
\subsubsection{e) $\{(0,0),(0,1),(0,2),(1,0),(1,1),(1,2),(2,0), (2, 2), (3, 3)\}$}
Reflexive, yes. Anti-symmetric, no. Transitive, no. Not a partial ordering.


%----------------------------------------------------------------------------------------
%----------------------------------------------------------------------------------------

\subsection{5. Which of these are posets?}
\subsubsection{a) $(\mathbb{Z},=)$}
Reflexive? $x=x$, yes. Anti-symmetric? $x=y$,$y=x$, yes. Transitive? $x=y$,$y=z$,$x=z$, yes. Therefore, yes this is a poset.
\subsubsection{b) $(\mathbb{Z},\ne )$}
Reflexive, $x \ne x $ No. Anti-symmetric, $x \ne y, y \ne x$, can't really tell anything definite, not anti-symmetric. Same reason not transitive. Not a poset.
\subsubsection{c) $(Z,\geq)$}
Reflexive? $x \geq x$ Yes. Anti-symmetric? $x \geq y$,$y \geq x$, yes. Transitive? $x \geq y, y \geq z, x \geq z$, yes. Yes, this is a poset.
\subsubsection{d) $(Z,\nmid)$}
Reflexive? No, since $x \nmid x $ is false. Right away, not a poset.


%----------------------------------------------------------------------------------------

%----------------------------------------------------------------------------------------

%----------------------------------------------------------------------------------------


%----------------------------------------------------------------------------------------


%----------------------------------------------------------------------------------------




%----------------------------------------------------------------------------------------


\end{document}
