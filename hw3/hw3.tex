%%%%%%%%%%%%%%%%%%%%%%%%%%%%%%%%%%%%%%%%%
% Sectioned Assignment
% LaTeX Template
% Version 2.0 (Sept. 26 2017)
%
% Heavily modified for use by:
% Nicholas Christiny (http://nchristiny.com/)
% Original author:
% Frits Wenneker (http://www.howtotex.com)
%
% License:
% CC BY-NC-SA 3.0 (http://creativecommons.org/licenses/by-nc-sa/3.0/)
%
%%%%%%%%%%%%%%%%%%%%%%%%%%%%%%%%%%%%%%%%%

%----------------------------------------------------------------------------------------
%	PACKAGES AND OTHER DOCUMENT CONFIGURATIONS
%----------------------------------------------------------------------------------------

\documentclass[11pt, oneside]{article} % A4 paper and 11pt font size

\usepackage[T1]{fontenc} % Use 8-bit encoding that has 256 glyphs
\usepackage{fourier} % Use the Adobe Utopia font for the document - comment this line to return to the LaTeX default
%\usepackage{fontspec}
%\setmainfont{Brill} % Using Brill http://www.brill.com/about/brill-fonts !!requires LuaTeX!! for error "cannot-use-pdftex", make sure to use LuaTeX (or XeTeX)
\usepackage[english]{babel} % English language/hyphenation
\usepackage{amsmath,amsfonts,amsthm,geometry,amssymb} % Math packages

%\usepackage{sectsty} % Allows customizing section commands
%\allsectionsfont{\scshape} % Make all sections small caps

\usepackage{fancyhdr} % Custom headers and footers
\pagestyle{fancyplain} % Makes all pages in the document conform to the custom headers and footers
\fancyhead{} % No page header - if you want one, create it in the same way as the footers below
\fancyfoot[L]{{\footnotesize{Math146 HW2}}} % Empty left footer
\fancyfoot[C]{} % Empty center footer
\fancyfoot[R]{Page \thepage} % Page numbering for right footer
\renewcommand{\headrulewidth}{0pt} % Remove header underlines
\renewcommand{\footrulewidth}{0pt} % Remove footer underlines
\setlength{\headheight}{13.6pt} % Customize the height of the header

\numberwithin{equation}{section} % Number equations within sections (i.e. 1.1, 1.2, 2.1, 2.2 instead of 1, 2, 3, 4)
\numberwithin{figure}{section} % Number figures within sections (i.e. 1.1, 1.2, 2.1, 2.2 instead of 1, 2, 3, 4)
\numberwithin{table}{section} % Number tables within sections (i.e. 1.1, 1.2, 2.1, 2.2 instead of 1, 2, 3, 4)

%\setlength\parindent{0pt} % Removes all indentation from paragraphs - comment this line for an assignment with lots of text
\setcounter{secnumdepth}{0} % Suppress section numbering
%----------------------------------------------------------------------------------------
%	TITLE SECTION
%----------------------------------------------------------------------------------------

\newcommand{\horrule}[1]{\rule{\linewidth}{#1}} % Create horizontal rule command with 1 argument of height

\title{
\normalfont \normalsize
\textsc{Wilbur Wright College, Fall 2017} \\ [25pt] % Your university, school and/or department name(s)
\horrule{0.5pt} \\[0.4cm] % Thin top horizontal rule
\huge Discrete Mathematics Homework III \\
\huge Mathematical Induction and Combinations \\ % The assignment title
\horrule{2pt} \\[0.5cm] % Thick bottom horizontal rule
}

\author{Nicholas Christiny} % Your name
\date{\normalsize\today} % Today's date or a custom date

\begin{document}

\maketitle % Print the title

%----------------------------------------------------------------------------------------
%----------------------------------------------------------------------------------------
%----------------------------------------------------------------------------------------

\section{\textsection5.1}
\subsection{3. Let $P(n)$ be the statement that $1^2 + 2^2 +...+n^2 = n(n + 1)(2n + 1)/6$ for the positive integer n.}
\subsubsection{a) What is the statement $P(1)$?}
Base case.
\subsubsection{b) Show that $P(1)$ is true, completing the basis step of the proof.}
\subsubsection{c) What is the inductive hypothesis?}
\subsubsection{d) What do you need to prove in the inductive step?}
\subsubsection{e) Complete the inductive step, identifying where you use the inductive hypothesis.}
\subsubsection{f) Explain why these steps show that this formula is true whenever $n$ is a positive integer.}

\subsection{5. Prove that $1^2 +3^2 +5^2 +...+(2n+1)^2 =(n+1) (2n + 1)(2n + 3)/3$ whenever $n$ is a nonnegative integer.}
%----------------------------------------------------------------------------------------
\subsection{7. Prove that $3+3\cdot 5+3 \cdot 5^2+...+3 \cdot 5^n=3(5^{n+1} -1)/4$ whenever $n$ is a nonnegative integer.
}

%----------------------------------------------------------------------------------------

\subsection{21. Prove that $2^n > n^2$ if $n$ is an integer greater than $4$.}

%----------------------------------------------------------------------------------------
%----------------------------------------------------------------------------------------

\subsection{31. Prove that $2$ divides $n^2 + n$ whenever $n$ is a positive integer.}

%----------------------------------------------------------------------------------------
\subsection{39. Prove that if $A_1,A_2,...,A_n$ and $B_1,B_2,...,B_n$ are sets such that $A_j \subseteq B_j$ for $j = 1,2,...,n,$ then}
\begin{displaymath}
 \displaystyle\bigcap^n_{j = 1} A_j \subseteq \bigcap^n_{j = 1} B_j.
\end{displaymath}

\subsection{43. Prove that if $A_1, A_2,..., A_n$ are subsets of a universal set $U$ , then}
\begin{displaymath}
 \displaystyle\bigcup^n_{k = 1} A_k = \bigcap^n_{k = 1} \overline{A_k}.
\end{displaymath}
%----------------------------------------------------------------------------------------

\end{document}
